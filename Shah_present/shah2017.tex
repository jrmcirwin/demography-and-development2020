\documentclass[ignorenonframetext,]{beamer}
\setbeamertemplate{caption}[numbered]
\setbeamertemplate{caption label separator}{: }
\setbeamercolor{caption name}{fg=normal text.fg}
\beamertemplatenavigationsymbolsempty
\usepackage{lmodern}
\usepackage{amssymb,amsmath}
\usepackage{ifxetex,ifluatex}
\usepackage{fixltx2e} % provides \textsubscript
\ifnum 0\ifxetex 1\fi\ifluatex 1\fi=0 % if pdftex
  \usepackage[T1]{fontenc}
  \usepackage[utf8]{inputenc}
\else % if luatex or xelatex
  \ifxetex
    \usepackage{mathspec}
  \else
    \usepackage{fontspec}
  \fi
  \defaultfontfeatures{Ligatures=TeX,Scale=MatchLowercase}
\fi
% use upquote if available, for straight quotes in verbatim environments
\IfFileExists{upquote.sty}{\usepackage{upquote}}{}
% use microtype if available
\IfFileExists{microtype.sty}{%
\usepackage{microtype}
\UseMicrotypeSet[protrusion]{basicmath} % disable protrusion for tt fonts
}{}
\newif\ifbibliography
\hypersetup{
            pdftitle={Drought of Opportunities},
            pdfauthor={Jesse McDevitt-Irwin},
            pdfborder={0 0 0},
            breaklinks=true}
\urlstyle{same}  % don't use monospace font for urls
\usepackage{longtable,booktabs}
\usepackage{caption}
% These lines are needed to make table captions work with longtable:
\makeatletter
\def\fnum@table{\tablename~\thetable}
\makeatother
\usepackage{graphicx,grffile}
\makeatletter
\def\maxwidth{\ifdim\Gin@nat@width>\linewidth\linewidth\else\Gin@nat@width\fi}
\def\maxheight{\ifdim\Gin@nat@height>\textheight0.8\textheight\else\Gin@nat@height\fi}
\makeatother
% Scale images if necessary, so that they will not overflow the page
% margins by default, and it is still possible to overwrite the defaults
% using explicit options in \includegraphics[width, height, ...]{}
\setkeys{Gin}{width=\maxwidth,height=\maxheight,keepaspectratio}

% Prevent slide breaks in the middle of a paragraph:
\widowpenalties 1 10000
\raggedbottom

\AtBeginPart{
  \let\insertpartnumber\relax
  \let\partname\relax
  \frame{\partpage}
}
\AtBeginSection{
  \ifbibliography
  \else
    \let\insertsectionnumber\relax
    \let\sectionname\relax
    \frame{\sectionpage}
  \fi
}
\AtBeginSubsection{
  \let\insertsubsectionnumber\relax
  \let\subsectionname\relax
  \frame{\subsectionpage}
}

\setlength{\parindent}{0pt}
\setlength{\parskip}{6pt plus 2pt minus 1pt}
\setlength{\emergencystretch}{3em}  % prevent overfull lines
\providecommand{\tightlist}{%
  \setlength{\itemsep}{0pt}\setlength{\parskip}{0pt}}
\setcounter{secnumdepth}{0}

\title{Drought of Opportunities}
\subtitle{Manisha Shah and Bryce Steinburg, Journal of Political Economy (2017)}
\author{Jesse McDevitt-Irwin}
\date{13 February, 2020}

\begin{document}
\frame{\titlepage}

\begin{frame}{\# Summary}

\begin{itemize}[<+->]
\tightlist
\item
  The authors explore the ambiguous effect of wages on human capital
  formation in children, using rainfall as an instrument.
\item
  They find that exposure to rainfall ``shocks'' in early childhood
  leads to an increase in human capital.
\item
  For exposure to rainfall shocks after the age of 5, they find the
  opposite effect: rainfall leads to a decrease in human capital
\end{itemize}

\end{frame}

\begin{frame}{\includegraphics{shah2017_files/figure-beamer/authors model-1.pdf}}

\begin{block}{Model}

\[ U(c_1,c_2,c_3) = u_1(c_q)+ \beta u_2(c_2) + \beta^2 u(e_3) \]
\[ c_1=w_1 h \\
 c_2 = w_2[h+ (1-s_2)e_2]\]

\[ e_1 = 0 \\
e_2 = f_2(c_1) \\
e_3 = f_3(e_2,c_2,s_2)
\]

No choices in period one, restrict attention to periods 2 and 3. These
can be thought of as early childhood, and post age 5.

\begin{longtable}[]{@{}l@{}}
\toprule
\begin{minipage}[t]{0.05\columnwidth}\raggedright\strut
\includegraphics{shah2017_files/figure-beamer/reality-1.pdf}\strut
\end{minipage}\tabularnewline
\bottomrule
\end{longtable}

\end{block}

\begin{block}{Effect of School-age Wages on Human Capital}

The authors solve for the effect of wages on schooling in an interior
solution:

(1)∂s2*∂w2∝−e2(∂u2∂c2+β∂f3∂c2)︷Substitution Effect (−)−{[}h+(1−s2*)e2{]}w2e2∂2u2∂c22︷Income Effect (+)+{[}h+(1−s2*)e2{]}β∂Θ∂c2︷Effect of c2 on Net Impact of Schooling.

\textless{}

We see that the net effect of school-age wages on schooling is
ambiguous.

\end{block}

\end{frame}

\begin{frame}

\begin{block}{Data}

\begin{itemize}[<+->]
\tightlist
\item
  The authors take the distribution of rainfall by district, and
  classify rainfall (drought) ``shocks'' as district years which place
  in the top (bottom) quintile of the district-distribution.
\end{itemize}

\begin{itemize}[<+->]
\tightlist
\item
  Data on cognitive tests, 2005-09, for 2 million children aged 5-16.
  Includes children not enrolled at school. Scores are constructed as
  the sum of correct answers on each test.
\end{itemize}

\begin{itemize}[<+->]
\tightlist
\item
  Data from 2004-08 on wages and employment, merged with rainfall on the
  district level. Rainfall is positively related to wages.
\end{itemize}

\end{block}

\end{frame}

\begin{frame}

\begin{block}{Empirical Specification}

\protect\hypertarget{df18}{}{(5)Sijty=α+β1δj,y+β2δj,y−1+ζθj,t+γj+ɸt+ψy+ϵijty,}where
\(S_{ijty}\) is the measure of human capital or schooling for student
\(i\) in district \(j\) born in year \(t\) and surveyed in year \(y\).
\(\theta\) is a vector of early-childhood rainfall shocks.

\end{block}

\end{frame}

\begin{frame}

\begin{block}{Results}

\begin{figure}
\includegraphics[width=18.07in]{/home/friend/Téléchargements/tb2.1} \caption{ }\label{fig:image-ref-for-in-text}
\end{figure}

\end{block}

\end{frame}

\end{document}
